% Generated by Sphinx.
\def\sphinxdocclass{report}
\documentclass[letterpaper,10pt,english]{sphinxmanual}
\usepackage[utf8]{inputenc}
\DeclareUnicodeCharacter{00A0}{\nobreakspace}
\usepackage{cmap}
\usepackage[T1]{fontenc}
\usepackage{babel}
\usepackage{times}
\usepackage[Bjarne]{fncychap}
\usepackage{longtable}
\usepackage{sphinx}
\usepackage{multirow}


\title{cv\_kickstarter Documentation}
\date{December 04, 2014}
\release{}
\author{Author}
\newcommand{\sphinxlogo}{}
\renewcommand{\releasename}{Release}
\makeindex

\makeatletter
\def\PYG@reset{\let\PYG@it=\relax \let\PYG@bf=\relax%
    \let\PYG@ul=\relax \let\PYG@tc=\relax%
    \let\PYG@bc=\relax \let\PYG@ff=\relax}
\def\PYG@tok#1{\csname PYG@tok@#1\endcsname}
\def\PYG@toks#1+{\ifx\relax#1\empty\else%
    \PYG@tok{#1}\expandafter\PYG@toks\fi}
\def\PYG@do#1{\PYG@bc{\PYG@tc{\PYG@ul{%
    \PYG@it{\PYG@bf{\PYG@ff{#1}}}}}}}
\def\PYG#1#2{\PYG@reset\PYG@toks#1+\relax+\PYG@do{#2}}

\expandafter\def\csname PYG@tok@gd\endcsname{\def\PYG@tc##1{\textcolor[rgb]{0.63,0.00,0.00}{##1}}}
\expandafter\def\csname PYG@tok@gu\endcsname{\let\PYG@bf=\textbf\def\PYG@tc##1{\textcolor[rgb]{0.50,0.00,0.50}{##1}}}
\expandafter\def\csname PYG@tok@gt\endcsname{\def\PYG@tc##1{\textcolor[rgb]{0.00,0.27,0.87}{##1}}}
\expandafter\def\csname PYG@tok@gs\endcsname{\let\PYG@bf=\textbf}
\expandafter\def\csname PYG@tok@gr\endcsname{\def\PYG@tc##1{\textcolor[rgb]{1.00,0.00,0.00}{##1}}}
\expandafter\def\csname PYG@tok@cm\endcsname{\let\PYG@it=\textit\def\PYG@tc##1{\textcolor[rgb]{0.25,0.50,0.56}{##1}}}
\expandafter\def\csname PYG@tok@vg\endcsname{\def\PYG@tc##1{\textcolor[rgb]{0.73,0.38,0.84}{##1}}}
\expandafter\def\csname PYG@tok@m\endcsname{\def\PYG@tc##1{\textcolor[rgb]{0.13,0.50,0.31}{##1}}}
\expandafter\def\csname PYG@tok@mh\endcsname{\def\PYG@tc##1{\textcolor[rgb]{0.13,0.50,0.31}{##1}}}
\expandafter\def\csname PYG@tok@cs\endcsname{\def\PYG@tc##1{\textcolor[rgb]{0.25,0.50,0.56}{##1}}\def\PYG@bc##1{\setlength{\fboxsep}{0pt}\colorbox[rgb]{1.00,0.94,0.94}{\strut ##1}}}
\expandafter\def\csname PYG@tok@ge\endcsname{\let\PYG@it=\textit}
\expandafter\def\csname PYG@tok@vc\endcsname{\def\PYG@tc##1{\textcolor[rgb]{0.73,0.38,0.84}{##1}}}
\expandafter\def\csname PYG@tok@il\endcsname{\def\PYG@tc##1{\textcolor[rgb]{0.13,0.50,0.31}{##1}}}
\expandafter\def\csname PYG@tok@go\endcsname{\def\PYG@tc##1{\textcolor[rgb]{0.20,0.20,0.20}{##1}}}
\expandafter\def\csname PYG@tok@cp\endcsname{\def\PYG@tc##1{\textcolor[rgb]{0.00,0.44,0.13}{##1}}}
\expandafter\def\csname PYG@tok@gi\endcsname{\def\PYG@tc##1{\textcolor[rgb]{0.00,0.63,0.00}{##1}}}
\expandafter\def\csname PYG@tok@gh\endcsname{\let\PYG@bf=\textbf\def\PYG@tc##1{\textcolor[rgb]{0.00,0.00,0.50}{##1}}}
\expandafter\def\csname PYG@tok@ni\endcsname{\let\PYG@bf=\textbf\def\PYG@tc##1{\textcolor[rgb]{0.84,0.33,0.22}{##1}}}
\expandafter\def\csname PYG@tok@nl\endcsname{\let\PYG@bf=\textbf\def\PYG@tc##1{\textcolor[rgb]{0.00,0.13,0.44}{##1}}}
\expandafter\def\csname PYG@tok@nn\endcsname{\let\PYG@bf=\textbf\def\PYG@tc##1{\textcolor[rgb]{0.05,0.52,0.71}{##1}}}
\expandafter\def\csname PYG@tok@no\endcsname{\def\PYG@tc##1{\textcolor[rgb]{0.38,0.68,0.84}{##1}}}
\expandafter\def\csname PYG@tok@na\endcsname{\def\PYG@tc##1{\textcolor[rgb]{0.25,0.44,0.63}{##1}}}
\expandafter\def\csname PYG@tok@nb\endcsname{\def\PYG@tc##1{\textcolor[rgb]{0.00,0.44,0.13}{##1}}}
\expandafter\def\csname PYG@tok@nc\endcsname{\let\PYG@bf=\textbf\def\PYG@tc##1{\textcolor[rgb]{0.05,0.52,0.71}{##1}}}
\expandafter\def\csname PYG@tok@nd\endcsname{\let\PYG@bf=\textbf\def\PYG@tc##1{\textcolor[rgb]{0.33,0.33,0.33}{##1}}}
\expandafter\def\csname PYG@tok@ne\endcsname{\def\PYG@tc##1{\textcolor[rgb]{0.00,0.44,0.13}{##1}}}
\expandafter\def\csname PYG@tok@nf\endcsname{\def\PYG@tc##1{\textcolor[rgb]{0.02,0.16,0.49}{##1}}}
\expandafter\def\csname PYG@tok@si\endcsname{\let\PYG@it=\textit\def\PYG@tc##1{\textcolor[rgb]{0.44,0.63,0.82}{##1}}}
\expandafter\def\csname PYG@tok@s2\endcsname{\def\PYG@tc##1{\textcolor[rgb]{0.25,0.44,0.63}{##1}}}
\expandafter\def\csname PYG@tok@vi\endcsname{\def\PYG@tc##1{\textcolor[rgb]{0.73,0.38,0.84}{##1}}}
\expandafter\def\csname PYG@tok@nt\endcsname{\let\PYG@bf=\textbf\def\PYG@tc##1{\textcolor[rgb]{0.02,0.16,0.45}{##1}}}
\expandafter\def\csname PYG@tok@nv\endcsname{\def\PYG@tc##1{\textcolor[rgb]{0.73,0.38,0.84}{##1}}}
\expandafter\def\csname PYG@tok@s1\endcsname{\def\PYG@tc##1{\textcolor[rgb]{0.25,0.44,0.63}{##1}}}
\expandafter\def\csname PYG@tok@gp\endcsname{\let\PYG@bf=\textbf\def\PYG@tc##1{\textcolor[rgb]{0.78,0.36,0.04}{##1}}}
\expandafter\def\csname PYG@tok@sh\endcsname{\def\PYG@tc##1{\textcolor[rgb]{0.25,0.44,0.63}{##1}}}
\expandafter\def\csname PYG@tok@ow\endcsname{\let\PYG@bf=\textbf\def\PYG@tc##1{\textcolor[rgb]{0.00,0.44,0.13}{##1}}}
\expandafter\def\csname PYG@tok@sx\endcsname{\def\PYG@tc##1{\textcolor[rgb]{0.78,0.36,0.04}{##1}}}
\expandafter\def\csname PYG@tok@bp\endcsname{\def\PYG@tc##1{\textcolor[rgb]{0.00,0.44,0.13}{##1}}}
\expandafter\def\csname PYG@tok@c1\endcsname{\let\PYG@it=\textit\def\PYG@tc##1{\textcolor[rgb]{0.25,0.50,0.56}{##1}}}
\expandafter\def\csname PYG@tok@kc\endcsname{\let\PYG@bf=\textbf\def\PYG@tc##1{\textcolor[rgb]{0.00,0.44,0.13}{##1}}}
\expandafter\def\csname PYG@tok@c\endcsname{\let\PYG@it=\textit\def\PYG@tc##1{\textcolor[rgb]{0.25,0.50,0.56}{##1}}}
\expandafter\def\csname PYG@tok@mf\endcsname{\def\PYG@tc##1{\textcolor[rgb]{0.13,0.50,0.31}{##1}}}
\expandafter\def\csname PYG@tok@err\endcsname{\def\PYG@bc##1{\setlength{\fboxsep}{0pt}\fcolorbox[rgb]{1.00,0.00,0.00}{1,1,1}{\strut ##1}}}
\expandafter\def\csname PYG@tok@mb\endcsname{\def\PYG@tc##1{\textcolor[rgb]{0.13,0.50,0.31}{##1}}}
\expandafter\def\csname PYG@tok@ss\endcsname{\def\PYG@tc##1{\textcolor[rgb]{0.32,0.47,0.09}{##1}}}
\expandafter\def\csname PYG@tok@sr\endcsname{\def\PYG@tc##1{\textcolor[rgb]{0.14,0.33,0.53}{##1}}}
\expandafter\def\csname PYG@tok@mo\endcsname{\def\PYG@tc##1{\textcolor[rgb]{0.13,0.50,0.31}{##1}}}
\expandafter\def\csname PYG@tok@kd\endcsname{\let\PYG@bf=\textbf\def\PYG@tc##1{\textcolor[rgb]{0.00,0.44,0.13}{##1}}}
\expandafter\def\csname PYG@tok@mi\endcsname{\def\PYG@tc##1{\textcolor[rgb]{0.13,0.50,0.31}{##1}}}
\expandafter\def\csname PYG@tok@kn\endcsname{\let\PYG@bf=\textbf\def\PYG@tc##1{\textcolor[rgb]{0.00,0.44,0.13}{##1}}}
\expandafter\def\csname PYG@tok@o\endcsname{\def\PYG@tc##1{\textcolor[rgb]{0.40,0.40,0.40}{##1}}}
\expandafter\def\csname PYG@tok@kr\endcsname{\let\PYG@bf=\textbf\def\PYG@tc##1{\textcolor[rgb]{0.00,0.44,0.13}{##1}}}
\expandafter\def\csname PYG@tok@s\endcsname{\def\PYG@tc##1{\textcolor[rgb]{0.25,0.44,0.63}{##1}}}
\expandafter\def\csname PYG@tok@kp\endcsname{\def\PYG@tc##1{\textcolor[rgb]{0.00,0.44,0.13}{##1}}}
\expandafter\def\csname PYG@tok@w\endcsname{\def\PYG@tc##1{\textcolor[rgb]{0.73,0.73,0.73}{##1}}}
\expandafter\def\csname PYG@tok@kt\endcsname{\def\PYG@tc##1{\textcolor[rgb]{0.56,0.13,0.00}{##1}}}
\expandafter\def\csname PYG@tok@sc\endcsname{\def\PYG@tc##1{\textcolor[rgb]{0.25,0.44,0.63}{##1}}}
\expandafter\def\csname PYG@tok@sb\endcsname{\def\PYG@tc##1{\textcolor[rgb]{0.25,0.44,0.63}{##1}}}
\expandafter\def\csname PYG@tok@k\endcsname{\let\PYG@bf=\textbf\def\PYG@tc##1{\textcolor[rgb]{0.00,0.44,0.13}{##1}}}
\expandafter\def\csname PYG@tok@se\endcsname{\let\PYG@bf=\textbf\def\PYG@tc##1{\textcolor[rgb]{0.25,0.44,0.63}{##1}}}
\expandafter\def\csname PYG@tok@sd\endcsname{\let\PYG@it=\textit\def\PYG@tc##1{\textcolor[rgb]{0.25,0.44,0.63}{##1}}}

\def\PYGZbs{\char`\\}
\def\PYGZus{\char`\_}
\def\PYGZob{\char`\{}
\def\PYGZcb{\char`\}}
\def\PYGZca{\char`\^}
\def\PYGZam{\char`\&}
\def\PYGZlt{\char`\<}
\def\PYGZgt{\char`\>}
\def\PYGZsh{\char`\#}
\def\PYGZpc{\char`\%}
\def\PYGZdl{\char`\$}
\def\PYGZhy{\char`\-}
\def\PYGZsq{\char`\'}
\def\PYGZdq{\char`\"}
\def\PYGZti{\char`\~}
% for compatibility with earlier versions
\def\PYGZat{@}
\def\PYGZlb{[}
\def\PYGZrb{]}
\makeatother

\renewcommand\PYGZsq{\textquotesingle}

\begin{document}

\maketitle
\tableofcontents
\phantomsection\label{index::doc}


Contents:


\chapter{cv\_kickstarter package}
\label{cv_kickstarter:welcome-to-cv-kickstarter-s-documentation}\label{cv_kickstarter:cv-kickstarter-package}\label{cv_kickstarter::doc}

\section{Subpackages}
\label{cv_kickstarter:subpackages}

\subsection{cv\_kickstarter.models package}
\label{cv_kickstarter.models:cv-kickstarter-models-package}\label{cv_kickstarter.models::doc}

\subsubsection{Submodules}
\label{cv_kickstarter.models:submodules}

\subsubsection{cv\_kickstarter.models.exam\_result\_programme module}
\label{cv_kickstarter.models:module-cv_kickstarter.models.exam_result_programme}\label{cv_kickstarter.models:cv-kickstarter-models-exam-result-programme-module}\index{cv\_kickstarter.models.exam\_result\_programme (module)}
The exam result programme with exam results for a given programme.
\index{ExamResultProgramme (class in cv\_kickstarter.models.exam\_result\_programme)}

\begin{fulllineitems}
\phantomsection\label{cv_kickstarter.models:cv_kickstarter.models.exam_result_programme.ExamResultProgramme}\pysiglinewithargsret{\strong{class }\code{cv\_kickstarter.models.exam\_result\_programme.}\bfcode{ExamResultProgramme}}{\emph{name}, \emph{passed\_ects}, \emph{total\_ects}, \emph{exam\_results}}{}
Bases: \code{object}

Representing exam results for a given programme (bachelor, master..).
\index{average\_grade (cv\_kickstarter.models.exam\_result\_programme.ExamResultProgramme attribute)}

\begin{fulllineitems}
\phantomsection\label{cv_kickstarter.models:cv_kickstarter.models.exam_result_programme.ExamResultProgramme.average_grade}\pysigline{\bfcode{average\_grade}}
Return the average grade of the student.

\end{fulllineitems}

\index{is\_done (cv\_kickstarter.models.exam\_result\_programme.ExamResultProgramme attribute)}

\begin{fulllineitems}
\phantomsection\label{cv_kickstarter.models:cv_kickstarter.models.exam_result_programme.ExamResultProgramme.is_done}\pysigline{\bfcode{is\_done}}
Return true if all the courses are passed for the programme.

\end{fulllineitems}


\end{fulllineitems}



\subsubsection{cv\_kickstarter.models.user\_cv module}
\label{cv_kickstarter.models:module-cv_kickstarter.models.user_cv}\label{cv_kickstarter.models:cv-kickstarter-models-user-cv-module}\index{cv\_kickstarter.models.user\_cv (module)}
UserCV for displaying the students CV on the frontend.
\index{UserCV (class in cv\_kickstarter.models.user\_cv)}

\begin{fulllineitems}
\phantomsection\label{cv_kickstarter.models:cv_kickstarter.models.user_cv.UserCV}\pysiglinewithargsret{\strong{class }\code{cv\_kickstarter.models.user\_cv.}\bfcode{UserCV}}{\emph{first\_name}, \emph{last\_name}, \emph{exam\_result\_programmes}, \emph{keywords}}{}
Bases: \code{object}

UserCV object works as an view object to be used in the view.
\index{course\_title\_sentence() (cv\_kickstarter.models.user\_cv.UserCV method)}

\begin{fulllineitems}
\phantomsection\label{cv_kickstarter.models:cv_kickstarter.models.user_cv.UserCV.course_title_sentence}\pysiglinewithargsret{\bfcode{course\_title\_sentence}}{\emph{keyword}}{}
Return a course title sentence.

If one course: 01234 Course
For two courses: 01234 Course 1 and 01235 Course 2
For three courses: 01234 C1, 01235 C2 and 012346 C3

\end{fulllineitems}

\index{full\_name (cv\_kickstarter.models.user\_cv.UserCV attribute)}

\begin{fulllineitems}
\phantomsection\label{cv_kickstarter.models:cv_kickstarter.models.user_cv.UserCV.full_name}\pysigline{\bfcode{full\_name}}
Return the full name of the student.

\end{fulllineitems}

\index{highest\_ranked\_keywords (cv\_kickstarter.models.user\_cv.UserCV attribute)}

\begin{fulllineitems}
\phantomsection\label{cv_kickstarter.models:cv_kickstarter.models.user_cv.UserCV.highest_ranked_keywords}\pysigline{\bfcode{highest\_ranked\_keywords}}
Return the 50 highest ranked keywords.

\end{fulllineitems}


\end{fulllineitems}



\subsubsection{cv\_kickstarter.models.user\_cv\_builder module}
\label{cv_kickstarter.models:module-cv_kickstarter.models.user_cv_builder}\label{cv_kickstarter.models:cv-kickstarter-models-user-cv-builder-module}\index{cv\_kickstarter.models.user\_cv\_builder (module)}
Builds a User CV to display on the frontend.
\index{CampusNetExamResultMapper (class in cv\_kickstarter.models.user\_cv\_builder)}

\begin{fulllineitems}
\phantomsection\label{cv_kickstarter.models:cv_kickstarter.models.user_cv_builder.CampusNetExamResultMapper}\pysiglinewithargsret{\strong{class }\code{cv\_kickstarter.models.user\_cv\_builder.}\bfcode{CampusNetExamResultMapper}}{\emph{cn\_exam\_result\_programme}}{}
Bases: \code{object}

Maps an exam result programme to an object ready for frontend.
\index{mapped\_exam\_result() (cv\_kickstarter.models.user\_cv\_builder.CampusNetExamResultMapper method)}

\begin{fulllineitems}
\phantomsection\label{cv_kickstarter.models:cv_kickstarter.models.user_cv_builder.CampusNetExamResultMapper.mapped_exam_result}\pysiglinewithargsret{\bfcode{mapped\_exam\_result}}{}{}
Return mapped exam result.

\end{fulllineitems}


\end{fulllineitems}

\index{UserCVBuilder (class in cv\_kickstarter.models.user\_cv\_builder)}

\begin{fulllineitems}
\phantomsection\label{cv_kickstarter.models:cv_kickstarter.models.user_cv_builder.UserCVBuilder}\pysiglinewithargsret{\strong{class }\code{cv\_kickstarter.models.user\_cv\_builder.}\bfcode{UserCVBuilder}}{\emph{campus\_net\_client}, \emph{mongo\_store}}{}
Bases: \code{object}

Builds a User CV to display on the frontend.
\index{build() (cv\_kickstarter.models.user\_cv\_builder.UserCVBuilder method)}

\begin{fulllineitems}
\phantomsection\label{cv_kickstarter.models:cv_kickstarter.models.user_cv_builder.UserCVBuilder.build}\pysiglinewithargsret{\bfcode{build}}{}{}
Return a UserCV.

\end{fulllineitems}

\index{grades (cv\_kickstarter.models.user\_cv\_builder.UserCVBuilder attribute)}

\begin{fulllineitems}
\phantomsection\label{cv_kickstarter.models:cv_kickstarter.models.user_cv_builder.UserCVBuilder.grades}\pysigline{\bfcode{grades}}
Return grades from campus\_net\_client.

\end{fulllineitems}

\index{user (cv\_kickstarter.models.user\_cv\_builder.UserCVBuilder attribute)}

\begin{fulllineitems}
\phantomsection\label{cv_kickstarter.models:cv_kickstarter.models.user_cv_builder.UserCVBuilder.user}\pysigline{\bfcode{user}}
Return a user with information from the campus\_net\_client.

\end{fulllineitems}


\end{fulllineitems}



\subsubsection{cv\_kickstarter.models.user\_cv\_dictionary\_mapper module}
\label{cv_kickstarter.models:cv-kickstarter-models-user-cv-dictionary-mapper-module}\label{cv_kickstarter.models:module-cv_kickstarter.models.user_cv_dictionary_mapper}\index{cv\_kickstarter.models.user\_cv\_dictionary\_mapper (module)}
Maps a UserCV into a dictionary.
\index{UserCVDictionaryMapper (class in cv\_kickstarter.models.user\_cv\_dictionary\_mapper)}

\begin{fulllineitems}
\phantomsection\label{cv_kickstarter.models:cv_kickstarter.models.user_cv_dictionary_mapper.UserCVDictionaryMapper}\pysigline{\strong{class }\code{cv\_kickstarter.models.user\_cv\_dictionary\_mapper.}\bfcode{UserCVDictionaryMapper}}
Bases: \code{object}

Maps a UserCV into a dictionary.
\index{user\_cv\_dict() (cv\_kickstarter.models.user\_cv\_dictionary\_mapper.UserCVDictionaryMapper method)}

\begin{fulllineitems}
\phantomsection\label{cv_kickstarter.models:cv_kickstarter.models.user_cv_dictionary_mapper.UserCVDictionaryMapper.user_cv_dict}\pysiglinewithargsret{\bfcode{user\_cv\_dict}}{\emph{user\_cv}, \emph{student\_number}}{}
Return a dictionary based on the UserCV.

\end{fulllineitems}


\end{fulllineitems}



\subsubsection{Module contents}
\label{cv_kickstarter.models:module-cv_kickstarter.models}\label{cv_kickstarter.models:module-contents}\index{cv\_kickstarter.models (module)}
The domain specific code for this project.


\section{Submodules}
\label{cv_kickstarter:submodules}

\section{cv\_kickstarter.academic\_skill\_set module}
\label{cv_kickstarter:module-cv_kickstarter.academic_skill_set}\label{cv_kickstarter:cv-kickstarter-academic-skill-set-module}\index{cv\_kickstarter.academic\_skill\_set (module)}
Extracts a skill set based on keywords in exam results.

AcademicSkillSet extracts a skill set based on keywords in the exam results.

The keywords are ranked by the freqency and grades.
\index{CourseKeyword (class in cv\_kickstarter.academic\_skill\_set)}

\begin{fulllineitems}
\phantomsection\label{cv_kickstarter:cv_kickstarter.academic_skill_set.CourseKeyword}\pysigline{\strong{class }\code{cv\_kickstarter.academic\_skill\_set.}\bfcode{CourseKeyword}}
Bases: \code{tuple}

CourseKeyword(keyword, rank, course\_numbers)
\index{course\_numbers (cv\_kickstarter.academic\_skill\_set.CourseKeyword attribute)}

\begin{fulllineitems}
\phantomsection\label{cv_kickstarter:cv_kickstarter.academic_skill_set.CourseKeyword.course_numbers}\pysigline{\bfcode{course\_numbers}}
Alias for field number 2

\end{fulllineitems}

\index{keyword (cv\_kickstarter.academic\_skill\_set.CourseKeyword attribute)}

\begin{fulllineitems}
\phantomsection\label{cv_kickstarter:cv_kickstarter.academic_skill_set.CourseKeyword.keyword}\pysigline{\bfcode{keyword}}
Alias for field number 0

\end{fulllineitems}

\index{rank (cv\_kickstarter.academic\_skill\_set.CourseKeyword attribute)}

\begin{fulllineitems}
\phantomsection\label{cv_kickstarter:cv_kickstarter.academic_skill_set.CourseKeyword.rank}\pysigline{\bfcode{rank}}
Alias for field number 1

\end{fulllineitems}


\end{fulllineitems}

\index{CourseSkillSet (class in cv\_kickstarter.academic\_skill\_set)}

\begin{fulllineitems}
\phantomsection\label{cv_kickstarter:cv_kickstarter.academic_skill_set.CourseSkillSet}\pysiglinewithargsret{\strong{class }\code{cv\_kickstarter.academic\_skill\_set.}\bfcode{CourseSkillSet}}{\emph{grade\_booster}, \emph{word\_scorer}}{}
Bases: \code{object}

Extract skill gained in a given course.
\index{skill\_set() (cv\_kickstarter.academic\_skill\_set.CourseSkillSet method)}

\begin{fulllineitems}
\phantomsection\label{cv_kickstarter:cv_kickstarter.academic_skill_set.CourseSkillSet.skill_set}\pysiglinewithargsret{\bfcode{skill\_set}}{\emph{tokenized\_exam\_result}}{}
Return a skill set gained in the given course.

\end{fulllineitems}


\end{fulllineitems}

\index{CourseSkillSetMerger (class in cv\_kickstarter.academic\_skill\_set)}

\begin{fulllineitems}
\phantomsection\label{cv_kickstarter:cv_kickstarter.academic_skill_set.CourseSkillSetMerger}\pysigline{\strong{class }\code{cv\_kickstarter.academic\_skill\_set.}\bfcode{CourseSkillSetMerger}}
Bases: \code{object}

Merges skill sets from several courses into a common skill set.
\index{student\_skill\_set() (cv\_kickstarter.academic\_skill\_set.CourseSkillSetMerger method)}

\begin{fulllineitems}
\phantomsection\label{cv_kickstarter:cv_kickstarter.academic_skill_set.CourseSkillSetMerger.student_skill_set}\pysiglinewithargsret{\bfcode{student\_skill\_set}}{\emph{passed\_courses\_skills}}{}
Return a raw skill set merged.

Each course skill set is merged, which means that common skills from
the different courses arge merged into one CourseKeyword, which
lists both courses.

For keywords contained in two course skill sets, the rank is summed as
the keyword gained in more courses are considered more important.

\end{fulllineitems}


\end{fulllineitems}

\index{KeywordGradeBooster (class in cv\_kickstarter.academic\_skill\_set)}

\begin{fulllineitems}
\phantomsection\label{cv_kickstarter:cv_kickstarter.academic_skill_set.KeywordGradeBooster}\pysiglinewithargsret{\strong{class }\code{cv\_kickstarter.academic\_skill\_set.}\bfcode{KeywordGradeBooster}}{\emph{average\_grade}}{}
Bases: \code{object}

Boosts the keyword rank with the grade.

Multiplies the given keyword rank score with a grade score from the course
the course originated.
\index{boosted\_keyword\_scores() (cv\_kickstarter.academic\_skill\_set.KeywordGradeBooster method)}

\begin{fulllineitems}
\phantomsection\label{cv_kickstarter:cv_kickstarter.academic_skill_set.KeywordGradeBooster.boosted_keyword_scores}\pysiglinewithargsret{\bfcode{boosted\_keyword\_scores}}{\emph{tokenized\_exam\_result}, \emph{scored\_keywords}}{}
Return a list of keywords with a rank boosted by grade.

\end{fulllineitems}


\end{fulllineitems}

\index{KeywordScoreCalculator (class in cv\_kickstarter.academic\_skill\_set)}

\begin{fulllineitems}
\phantomsection\label{cv_kickstarter:cv_kickstarter.academic_skill_set.KeywordScoreCalculator}\pysigline{\strong{class }\code{cv\_kickstarter.academic\_skill\_set.}\bfcode{KeywordScoreCalculator}}
Bases: \code{object}

``Calculates the raw keyword scores based on the word scores.
\index{keyword\_scorer() (cv\_kickstarter.academic\_skill\_set.KeywordScoreCalculator method)}

\begin{fulllineitems}
\phantomsection\label{cv_kickstarter:cv_kickstarter.academic_skill_set.KeywordScoreCalculator.keyword_scorer}\pysiglinewithargsret{\bfcode{keyword\_scorer}}{\emph{tokenized\_exam\_result}, \emph{word\_scorer}}{}
``Return keyword scores.

\end{fulllineitems}


\end{fulllineitems}

\index{KeywordScoreNormalizer (class in cv\_kickstarter.academic\_skill\_set)}

\begin{fulllineitems}
\phantomsection\label{cv_kickstarter:cv_kickstarter.academic_skill_set.KeywordScoreNormalizer}\pysigline{\strong{class }\code{cv\_kickstarter.academic\_skill\_set.}\bfcode{KeywordScoreNormalizer}}
Bases: \code{object}

Normalize the rank score of a keyword with average score in course.
\index{normalized\_keyword\_score() (cv\_kickstarter.academic\_skill\_set.KeywordScoreNormalizer method)}

\begin{fulllineitems}
\phantomsection\label{cv_kickstarter:cv_kickstarter.academic_skill_set.KeywordScoreNormalizer.normalized_keyword_score}\pysiglinewithargsret{\bfcode{normalized\_keyword\_score}}{\emph{keyword\_scorer}}{}
Return a list of keywords with rank divided by average rank.

\end{fulllineitems}


\end{fulllineitems}

\index{StudentSkillSet (class in cv\_kickstarter.academic\_skill\_set)}

\begin{fulllineitems}
\phantomsection\label{cv_kickstarter:cv_kickstarter.academic_skill_set.StudentSkillSet}\pysiglinewithargsret{\strong{class }\code{cv\_kickstarter.academic\_skill\_set.}\bfcode{StudentSkillSet}}{\emph{word\_scorer}, \emph{grade\_booster}, \emph{noice\_filter}}{}
Bases: \code{object}

Extract skill set of a student.
\index{skill\_set() (cv\_kickstarter.academic\_skill\_set.StudentSkillSet method)}

\begin{fulllineitems}
\phantomsection\label{cv_kickstarter:cv_kickstarter.academic_skill_set.StudentSkillSet.skill_set}\pysiglinewithargsret{\bfcode{skill\_set}}{\emph{tokenized\_exam\_results}}{}
Return a ranked skill set.

\end{fulllineitems}


\end{fulllineitems}

\index{StudentSkillSetNoiceFilter (class in cv\_kickstarter.academic\_skill\_set)}

\begin{fulllineitems}
\phantomsection\label{cv_kickstarter:cv_kickstarter.academic_skill_set.StudentSkillSetNoiceFilter}\pysiglinewithargsret{\strong{class }\code{cv\_kickstarter.academic\_skill\_set.}\bfcode{StudentSkillSetNoiceFilter}}{\emph{min\_keyword\_length}}{}
Bases: \code{object}

Filters away noice from the skill set.

Filters away noice meaning unwanted keywords in the skill set.
\index{filtered\_skill\_set() (cv\_kickstarter.academic\_skill\_set.StudentSkillSetNoiceFilter method)}

\begin{fulllineitems}
\phantomsection\label{cv_kickstarter:cv_kickstarter.academic_skill_set.StudentSkillSetNoiceFilter.filtered_skill_set}\pysiglinewithargsret{\bfcode{filtered\_skill\_set}}{\emph{course\_keywords}}{}
Return a filtered set of skills.

\end{fulllineitems}


\end{fulllineitems}

\index{WordFrequencyScoreCalculator (class in cv\_kickstarter.academic\_skill\_set)}

\begin{fulllineitems}
\phantomsection\label{cv_kickstarter:cv_kickstarter.academic_skill_set.WordFrequencyScoreCalculator}\pysigline{\strong{class }\code{cv\_kickstarter.academic\_skill\_set.}\bfcode{WordFrequencyScoreCalculator}}
Bases: \code{object}

Builds a word score dictionary based on word frequency.
\index{word\_scorer() (cv\_kickstarter.academic\_skill\_set.WordFrequencyScoreCalculator method)}

\begin{fulllineitems}
\phantomsection\label{cv_kickstarter:cv_kickstarter.academic_skill_set.WordFrequencyScoreCalculator.word_scorer}\pysiglinewithargsret{\bfcode{word\_scorer}}{\emph{tokenized\_course\_exam\_result}}{}
Return a dictionary of word scores.

The score for each word is the frequency of the word in all
exam results.

\end{fulllineitems}


\end{fulllineitems}

\index{skill\_set() (in module cv\_kickstarter.academic\_skill\_set)}

\begin{fulllineitems}
\phantomsection\label{cv_kickstarter:cv_kickstarter.academic_skill_set.skill_set}\pysiglinewithargsret{\code{cv\_kickstarter.academic\_skill\_set.}\bfcode{skill\_set}}{\emph{tokenized\_exam\_results}, \emph{min\_keyword\_length=4}}{}
Extract skill set based on tokenized exam results.

Tokenized exam results are exam results with tokens for each course.
\begin{description}
\item[{Optimal arguments are}] \leavevmode
min\_keyword\_length: The minimum amount of characters in a keyword

\end{description}

The last max\_keyword\_courses is used for filtering away keywords that
are too common in the courses (e.g. `course', `analysis').

\end{fulllineitems}



\section{cv\_kickstarter.cnapi module}
\label{cv_kickstarter:cv-kickstarter-cnapi-module}\label{cv_kickstarter:module-cv_kickstarter.cnapi}\index{cv\_kickstarter.cnapi (module)}
Python client for the CampusNet API.

This library is designed for having a nice interface for integrating with
the CampusNet API. The library provides an interface for the network requests
as well as objects for wrapping the data returned by the CampusNet API.

For documentation of the Campusnet API, see:

\href{https://www.campusnet.dtu.dk/data/Documentation/CampusNet\%20public\%20API.pdf}{https://www.campusnet.dtu.dk/data/Documentation/CampusNet\%20public\%20API.pdf}

Example of usage:

At first instantiate the api:

\begin{Verbatim}[commandchars=\\\{\}]
\PYG{g+gp}{\PYGZgt{}\PYGZgt{}\PYGZgt{} }\PYG{n}{app\PYGZus{}name} \PYG{o}{=} \PYG{l+s}{\PYGZsq{}}\PYG{l+s}{MyCampusNetApp}\PYG{l+s}{\PYGZsq{}}
\PYG{g+gp}{\PYGZgt{}\PYGZgt{}\PYGZgt{} }\PYG{n}{app\PYGZus{}token} \PYG{o}{=} \PYG{l+s}{\PYGZsq{}}\PYG{l+s}{sh2870272\PYGZhy{}2ush292\PYGZhy{}ji2u98s2\PYGZhy{}2h2821\PYGZhy{}jsw9j2ihs982}\PYG{l+s}{\PYGZsq{}}
\end{Verbatim}

\begin{Verbatim}[commandchars=\\\{\}]
\PYG{g+gp}{\PYGZgt{}\PYGZgt{}\PYGZgt{} }\PYG{n}{api} \PYG{o}{=} \PYG{n}{cnapi}\PYG{o}{.}\PYG{n}{CampusNetApi}\PYG{p}{(}\PYG{n}{app\PYGZus{}name}\PYG{p}{,} \PYG{n}{app\PYGZus{}token}\PYG{p}{)}
\end{Verbatim}

In order to fetch information, authenticate the student with the student number
and password:

\begin{Verbatim}[commandchars=\\\{\}]
\PYG{g+gp}{\PYGZgt{}\PYGZgt{}\PYGZgt{} }\PYG{n}{api}\PYG{o}{.}\PYG{n}{authenticate}\PYG{p}{(}\PYG{l+s}{\PYGZsq{}}\PYG{l+s}{s123456}\PYG{l+s}{\PYGZsq{}}\PYG{p}{,} \PYG{l+s}{\PYGZsq{}}\PYG{l+s}{secret\PYGZhy{}password}\PYG{l+s}{\PYGZsq{}}\PYG{p}{)}
\end{Verbatim}

or if the auth token is already in posession use:

\begin{Verbatim}[commandchars=\\\{\}]
\PYG{g+gp}{\PYGZgt{}\PYGZgt{}\PYGZgt{} }\PYG{n}{api}\PYG{o}{.}\PYG{n}{authenticate\PYGZus{}with\PYGZus{}token}\PYG{p}{(}\PYG{l+s}{\PYGZsq{}}\PYG{l+s}{s123456}\PYG{l+s}{\PYGZsq{}}\PYG{p}{,} \PYG{l+s}{\PYGZsq{}}\PYG{l+s}{21EF8196\PYGZhy{}ED05\PYGZhy{}4BAB\PYGZhy{}9081}\PYG{l+s}{\PYGZsq{}}\PYG{p}{)}
\end{Verbatim}

To fetch the grades of the given user:

\begin{Verbatim}[commandchars=\\\{\}]
\PYG{g+gp}{\PYGZgt{}\PYGZgt{}\PYGZgt{} }\PYG{n}{grades} \PYG{o}{=} \PYG{n}{api}\PYG{o}{.}\PYG{n}{grades}\PYG{p}{(}\PYG{p}{)}
\end{Verbatim}

To fetch the user infor of the given user:

\begin{Verbatim}[commandchars=\\\{\}]
\PYG{g+gp}{\PYGZgt{}\PYGZgt{}\PYGZgt{} }\PYG{n}{user} \PYG{o}{=} \PYG{n}{api}\PYG{o}{.}\PYG{n}{user}\PYG{p}{(}\PYG{p}{)}
\end{Verbatim}
\index{AbstractXmlInfoExtractor (class in cv\_kickstarter.cnapi)}

\begin{fulllineitems}
\phantomsection\label{cv_kickstarter:cv_kickstarter.cnapi.AbstractXmlInfoExtractor}\pysiglinewithargsret{\strong{class }\code{cv\_kickstarter.cnapi.}\bfcode{AbstractXmlInfoExtractor}}{\emph{response\_text}}{}
An abstract class for classes that extract information from xml.

The inheriting class needs to implement:
\begin{quote}

\_extract\_information
\end{quote}
\index{extract() (cv\_kickstarter.cnapi.AbstractXmlInfoExtractor method)}

\begin{fulllineitems}
\phantomsection\label{cv_kickstarter:cv_kickstarter.cnapi.AbstractXmlInfoExtractor.extract}\pysiglinewithargsret{\bfcode{extract}}{}{}
Extract the information from the given xml.

Returns a structure given by the child class implementing
`\_extract\_information'.

Returns None if the CampusNet API returns a Fault

\end{fulllineitems}


\end{fulllineitems}

\index{Authenticator (class in cv\_kickstarter.cnapi)}

\begin{fulllineitems}
\phantomsection\label{cv_kickstarter:cv_kickstarter.cnapi.Authenticator}\pysiglinewithargsret{\strong{class }\code{cv\_kickstarter.cnapi.}\bfcode{Authenticator}}{\emph{app\_name}, \emph{api\_token}}{}
Can authenticate a user based on a user name and a password.

The authentication fetches an auth\_token from the CampusNet API that
authenticates the user.

Example usage:

\begin{Verbatim}[commandchars=\\\{\}]
\PYG{g+gp}{\PYGZgt{}\PYGZgt{}\PYGZgt{} }\PYG{n}{auth} \PYG{o}{=} \PYG{n}{Authenticator}\PYG{p}{(}\PYG{l+s}{\PYGZsq{}}\PYG{l+s}{MyApp}\PYG{l+s}{\PYGZsq{}}\PYG{p}{,} \PYG{l+s}{\PYGZsq{}}\PYG{l+s}{app\PYGZhy{}token\PYGZhy{}123}\PYG{l+s}{\PYGZsq{}}\PYG{p}{)}
\PYG{g+gp}{\PYGZgt{}\PYGZgt{}\PYGZgt{} }\PYG{n}{token} \PYG{o}{=} \PYG{n}{auth}\PYG{o}{.}\PYG{n}{auth\PYGZus{}token}\PYG{p}{(}\PYG{l+s}{\PYGZsq{}}\PYG{l+s}{s1234}\PYG{l+s}{\PYGZsq{}}\PYG{p}{,} \PYG{l+s}{\PYGZsq{}}\PYG{l+s}{secretpass}\PYG{l+s}{\PYGZsq{}}\PYG{p}{)}
\end{Verbatim}
\index{auth\_token() (cv\_kickstarter.cnapi.Authenticator method)}

\begin{fulllineitems}
\phantomsection\label{cv_kickstarter:cv_kickstarter.cnapi.Authenticator.auth_token}\pysiglinewithargsret{\bfcode{auth\_token}}{\emph{username}, \emph{password}}{}
Fetch and return an authentication token from CampusNet API.

The request is a POST request send the given app name, app token,
username and password.

If the authentication fails, it will return None.

\end{fulllineitems}


\end{fulllineitems}

\index{CampusNetApi (class in cv\_kickstarter.cnapi)}

\begin{fulllineitems}
\phantomsection\label{cv_kickstarter:cv_kickstarter.cnapi.CampusNetApi}\pysiglinewithargsret{\strong{class }\code{cv\_kickstarter.cnapi.}\bfcode{CampusNetApi}}{\emph{app\_name}, \emph{api\_token}}{}
The interface class for the API.

This class acts as the top level interface of the api and wraps the
behaviour needed for fetching relevant information from from the API.
\index{authenticate() (cv\_kickstarter.cnapi.CampusNetApi method)}

\begin{fulllineitems}
\phantomsection\label{cv_kickstarter:cv_kickstarter.cnapi.CampusNetApi.authenticate}\pysiglinewithargsret{\bfcode{authenticate}}{\emph{student\_number}, \emph{password}}{}
Authenticate the given user by fetching an authentication token.

\end{fulllineitems}

\index{authenticate\_with\_token() (cv\_kickstarter.cnapi.CampusNetApi method)}

\begin{fulllineitems}
\phantomsection\label{cv_kickstarter:cv_kickstarter.cnapi.CampusNetApi.authenticate_with_token}\pysiglinewithargsret{\bfcode{authenticate\_with\_token}}{\emph{student\_number}, \emph{auth\_token}}{}
Authenticate the given user by the given authentication token.

\end{fulllineitems}

\index{grades() (cv\_kickstarter.cnapi.CampusNetApi method)}

\begin{fulllineitems}
\phantomsection\label{cv_kickstarter:cv_kickstarter.cnapi.CampusNetApi.grades}\pysiglinewithargsret{\bfcode{grades}}{}{}
Fetch the grades for the authenticated user.

\end{fulllineitems}

\index{is\_authenticated() (cv\_kickstarter.cnapi.CampusNetApi method)}

\begin{fulllineitems}
\phantomsection\label{cv_kickstarter:cv_kickstarter.cnapi.CampusNetApi.is_authenticated}\pysiglinewithargsret{\bfcode{is\_authenticated}}{}{}
Return a boolean indicating whether the user is authenticated.

\end{fulllineitems}

\index{user() (cv\_kickstarter.cnapi.CampusNetApi method)}

\begin{fulllineitems}
\phantomsection\label{cv_kickstarter:cv_kickstarter.cnapi.CampusNetApi.user}\pysiglinewithargsret{\bfcode{user}}{}{}
Fetch user infor for the authenticated user.

\end{fulllineitems}

\index{user\_picture() (cv\_kickstarter.cnapi.CampusNetApi method)}

\begin{fulllineitems}
\phantomsection\label{cv_kickstarter:cv_kickstarter.cnapi.CampusNetApi.user_picture}\pysiglinewithargsret{\bfcode{user\_picture}}{\emph{user\_id}}{}
``Fetch the user picture.

\end{fulllineitems}


\end{fulllineitems}

\index{ExamResult (class in cv\_kickstarter.cnapi)}

\begin{fulllineitems}
\phantomsection\label{cv_kickstarter:cv_kickstarter.cnapi.ExamResult}\pysigline{\strong{class }\code{cv\_kickstarter.cnapi.}\bfcode{ExamResult}}
Bases: \code{tuple}

ExamResult(course\_title, course\_number, ects\_points, grade, period, year)
\index{course\_number (cv\_kickstarter.cnapi.ExamResult attribute)}

\begin{fulllineitems}
\phantomsection\label{cv_kickstarter:cv_kickstarter.cnapi.ExamResult.course_number}\pysigline{\bfcode{course\_number}}
Alias for field number 1

\end{fulllineitems}

\index{course\_title (cv\_kickstarter.cnapi.ExamResult attribute)}

\begin{fulllineitems}
\phantomsection\label{cv_kickstarter:cv_kickstarter.cnapi.ExamResult.course_title}\pysigline{\bfcode{course\_title}}
Alias for field number 0

\end{fulllineitems}

\index{ects\_points (cv\_kickstarter.cnapi.ExamResult attribute)}

\begin{fulllineitems}
\phantomsection\label{cv_kickstarter:cv_kickstarter.cnapi.ExamResult.ects_points}\pysigline{\bfcode{ects\_points}}
Alias for field number 2

\end{fulllineitems}

\index{grade (cv\_kickstarter.cnapi.ExamResult attribute)}

\begin{fulllineitems}
\phantomsection\label{cv_kickstarter:cv_kickstarter.cnapi.ExamResult.grade}\pysigline{\bfcode{grade}}
Alias for field number 3

\end{fulllineitems}

\index{period (cv\_kickstarter.cnapi.ExamResult attribute)}

\begin{fulllineitems}
\phantomsection\label{cv_kickstarter:cv_kickstarter.cnapi.ExamResult.period}\pysigline{\bfcode{period}}
Alias for field number 4

\end{fulllineitems}

\index{year (cv\_kickstarter.cnapi.ExamResult attribute)}

\begin{fulllineitems}
\phantomsection\label{cv_kickstarter:cv_kickstarter.cnapi.ExamResult.year}\pysigline{\bfcode{year}}
Alias for field number 5

\end{fulllineitems}


\end{fulllineitems}

\index{ExamResultXmlMapper (class in cv\_kickstarter.cnapi)}

\begin{fulllineitems}
\phantomsection\label{cv_kickstarter:cv_kickstarter.cnapi.ExamResultXmlMapper}\pysiglinewithargsret{\strong{class }\code{cv\_kickstarter.cnapi.}\bfcode{ExamResultXmlMapper}}{\emph{exam\_result\_xml}}{}
Bases: \code{object}

Is able to extract and map exam result xml into ExamResult objects.
\index{exam\_result() (cv\_kickstarter.cnapi.ExamResultXmlMapper method)}

\begin{fulllineitems}
\phantomsection\label{cv_kickstarter:cv_kickstarter.cnapi.ExamResultXmlMapper.exam_result}\pysiglinewithargsret{\bfcode{exam\_result}}{}{}
Return an exam result object with information given by the xml.

\end{fulllineitems}


\end{fulllineitems}

\index{ProgramExamResults (class in cv\_kickstarter.cnapi)}

\begin{fulllineitems}
\phantomsection\label{cv_kickstarter:cv_kickstarter.cnapi.ProgramExamResults}\pysigline{\strong{class }\code{cv\_kickstarter.cnapi.}\bfcode{ProgramExamResults}}
Bases: \code{tuple}

ProgramExamResults(name, is\_active, passed\_ects\_points, exam\_results)
\index{exam\_results (cv\_kickstarter.cnapi.ProgramExamResults attribute)}

\begin{fulllineitems}
\phantomsection\label{cv_kickstarter:cv_kickstarter.cnapi.ProgramExamResults.exam_results}\pysigline{\bfcode{exam\_results}}
Alias for field number 3

\end{fulllineitems}

\index{is\_active (cv\_kickstarter.cnapi.ProgramExamResults attribute)}

\begin{fulllineitems}
\phantomsection\label{cv_kickstarter:cv_kickstarter.cnapi.ProgramExamResults.is_active}\pysigline{\bfcode{is\_active}}
Alias for field number 1

\end{fulllineitems}

\index{name (cv\_kickstarter.cnapi.ProgramExamResults attribute)}

\begin{fulllineitems}
\phantomsection\label{cv_kickstarter:cv_kickstarter.cnapi.ProgramExamResults.name}\pysigline{\bfcode{name}}
Alias for field number 0

\end{fulllineitems}

\index{passed\_ects\_points (cv\_kickstarter.cnapi.ProgramExamResults attribute)}

\begin{fulllineitems}
\phantomsection\label{cv_kickstarter:cv_kickstarter.cnapi.ProgramExamResults.passed_ects_points}\pysigline{\bfcode{passed\_ects\_points}}
Alias for field number 2

\end{fulllineitems}


\end{fulllineitems}

\index{Student (class in cv\_kickstarter.cnapi)}

\begin{fulllineitems}
\phantomsection\label{cv_kickstarter:cv_kickstarter.cnapi.Student}\pysigline{\strong{class }\code{cv\_kickstarter.cnapi.}\bfcode{Student}}
Bases: \code{tuple}

Student(first\_name, last\_name, email, user\_id)
\index{email (cv\_kickstarter.cnapi.Student attribute)}

\begin{fulllineitems}
\phantomsection\label{cv_kickstarter:cv_kickstarter.cnapi.Student.email}\pysigline{\bfcode{email}}
Alias for field number 2

\end{fulllineitems}

\index{first\_name (cv\_kickstarter.cnapi.Student attribute)}

\begin{fulllineitems}
\phantomsection\label{cv_kickstarter:cv_kickstarter.cnapi.Student.first_name}\pysigline{\bfcode{first\_name}}
Alias for field number 0

\end{fulllineitems}

\index{last\_name (cv\_kickstarter.cnapi.Student attribute)}

\begin{fulllineitems}
\phantomsection\label{cv_kickstarter:cv_kickstarter.cnapi.Student.last_name}\pysigline{\bfcode{last\_name}}
Alias for field number 1

\end{fulllineitems}

\index{user\_id (cv\_kickstarter.cnapi.Student attribute)}

\begin{fulllineitems}
\phantomsection\label{cv_kickstarter:cv_kickstarter.cnapi.Student.user_id}\pysigline{\bfcode{user\_id}}
Alias for field number 3

\end{fulllineitems}


\end{fulllineitems}

\index{UserClient (class in cv\_kickstarter.cnapi)}

\begin{fulllineitems}
\phantomsection\label{cv_kickstarter:cv_kickstarter.cnapi.UserClient}\pysiglinewithargsret{\strong{class }\code{cv\_kickstarter.cnapi.}\bfcode{UserClient}}{\emph{app\_name}, \emph{api\_token}, \emph{student\_number}, \emph{access\_token}}{}
Network client for fetching user information CampusNet API.
\index{get() (cv\_kickstarter.cnapi.UserClient method)}

\begin{fulllineitems}
\phantomsection\label{cv_kickstarter:cv_kickstarter.cnapi.UserClient.get}\pysiglinewithargsret{\bfcode{get}}{\emph{path}}{}
Perform a GET request to fetch information about the given user.

For example:
\begin{quote}

user\_client = UserClient(`MyApp', `api-token-123', `s123', `atoke')
user\_client.get(`Grades')
\end{quote}

will perform a GET request to:
\begin{quote}

\href{https://www.campusnet.dtu.dk/data/CurrentUser/Grades}{https://www.campusnet.dtu.dk/data/CurrentUser/Grades}
\end{quote}

with the headers:
\begin{quote}

X-appname: `MyApp'
X-token: `api-token-123'
accept-language: `da-DK'
X-Include-services-and-relations: `true'
\end{quote}

\end{fulllineitems}


\end{fulllineitems}

\index{UserGradesExtractor (class in cv\_kickstarter.cnapi)}

\begin{fulllineitems}
\phantomsection\label{cv_kickstarter:cv_kickstarter.cnapi.UserGradesExtractor}\pysiglinewithargsret{\strong{class }\code{cv\_kickstarter.cnapi.}\bfcode{UserGradesExtractor}}{\emph{response\_text}}{}
Bases: {\hyperref[cv_kickstarter:cv_kickstarter.cnapi.AbstractXmlInfoExtractor]{\code{cv\_kickstarter.cnapi.AbstractXmlInfoExtractor}}}

Is able to extract the grades of the given user.

\end{fulllineitems}

\index{UserInfoExtractor (class in cv\_kickstarter.cnapi)}

\begin{fulllineitems}
\phantomsection\label{cv_kickstarter:cv_kickstarter.cnapi.UserInfoExtractor}\pysiglinewithargsret{\strong{class }\code{cv\_kickstarter.cnapi.}\bfcode{UserInfoExtractor}}{\emph{response\_text}}{}
Bases: {\hyperref[cv_kickstarter:cv_kickstarter.cnapi.AbstractXmlInfoExtractor]{\code{cv\_kickstarter.cnapi.AbstractXmlInfoExtractor}}}

Is able to extract user info based on xml describe the user.

\end{fulllineitems}



\section{cv\_kickstarter.course\_keyword\_tokenizer module}
\label{cv_kickstarter:cv-kickstarter-course-keyword-tokenizer-module}\label{cv_kickstarter:module-cv_kickstarter.course_keyword_tokenizer}\index{cv\_kickstarter.course\_keyword\_tokenizer (module)}
Extract raw keywords from course.

course\_keyword\_tokenizer extracts raw keyword tokens from courses with
noun phrase chunking.
\index{CourseKeywordTokenizer (class in cv\_kickstarter.course\_keyword\_tokenizer)}

\begin{fulllineitems}
\phantomsection\label{cv_kickstarter:cv_kickstarter.course_keyword_tokenizer.CourseKeywordTokenizer}\pysiglinewithargsret{\strong{class }\code{cv\_kickstarter.course\_keyword\_tokenizer.}\bfcode{CourseKeywordTokenizer}}{\emph{course}}{}
Bases: \code{object}

Can extract keyword tokens from course.
\index{keyword\_tokens() (cv\_kickstarter.course\_keyword\_tokenizer.CourseKeywordTokenizer method)}

\begin{fulllineitems}
\phantomsection\label{cv_kickstarter:cv_kickstarter.course_keyword_tokenizer.CourseKeywordTokenizer.keyword_tokens}\pysiglinewithargsret{\bfcode{keyword\_tokens}}{}{}
Return keyword tokens for course.

\end{fulllineitems}


\end{fulllineitems}

\index{CourseSentenceExtractor (class in cv\_kickstarter.course\_keyword\_tokenizer)}

\begin{fulllineitems}
\phantomsection\label{cv_kickstarter:cv_kickstarter.course_keyword_tokenizer.CourseSentenceExtractor}\pysiglinewithargsret{\strong{class }\code{cv\_kickstarter.course\_keyword\_tokenizer.}\bfcode{CourseSentenceExtractor}}{\emph{course}}{}
Bases: \code{object}

Can extact sentences from course object.
\index{sentences() (cv\_kickstarter.course\_keyword\_tokenizer.CourseSentenceExtractor method)}

\begin{fulllineitems}
\phantomsection\label{cv_kickstarter:cv_kickstarter.course_keyword_tokenizer.CourseSentenceExtractor.sentences}\pysiglinewithargsret{\bfcode{sentences}}{}{}
Return sentences for course.

The sentences are based on title, contents, course\_objectives\_text
and course\_objectives

\end{fulllineitems}


\end{fulllineitems}

\index{TextKeywordChunkifier (class in cv\_kickstarter.course\_keyword\_tokenizer)}

\begin{fulllineitems}
\phantomsection\label{cv_kickstarter:cv_kickstarter.course_keyword_tokenizer.TextKeywordChunkifier}\pysigline{\strong{class }\code{cv\_kickstarter.course\_keyword\_tokenizer.}\bfcode{TextKeywordChunkifier}}
Bases: \code{object}

Can split text into several chunks by noun phrase chunking.

The chunks are extracted with noun phrase chunking with nouns and preceding
adjectives (if there are any).

The code in this class is inspired by the NLTK book and Finn Nielsens
code from Data Mining using Python 02819 from DTU.
\index{chunks() (cv\_kickstarter.course\_keyword\_tokenizer.TextKeywordChunkifier method)}

\begin{fulllineitems}
\phantomsection\label{cv_kickstarter:cv_kickstarter.course_keyword_tokenizer.TextKeywordChunkifier.chunks}\pysiglinewithargsret{\bfcode{chunks}}{\emph{text}}{}
Return keyword chunks from the given text.

\end{fulllineitems}


\end{fulllineitems}

\index{course\_keyword\_tokens() (in module cv\_kickstarter.course\_keyword\_tokenizer)}

\begin{fulllineitems}
\phantomsection\label{cv_kickstarter:cv_kickstarter.course_keyword_tokenizer.course_keyword_tokens}\pysiglinewithargsret{\code{cv\_kickstarter.course\_keyword\_tokenizer.}\bfcode{course\_keyword\_tokens}}{\emph{course}}{}
Return keyword tokens for course.

\end{fulllineitems}



\section{cv\_kickstarter.course\_repository module}
\label{cv_kickstarter:module-cv_kickstarter.course_repository}\label{cv_kickstarter:cv-kickstarter-course-repository-module}\index{cv\_kickstarter.course\_repository (module)}
Course Repository for storing and fetching courses from MongoDB.
\index{Course (class in cv\_kickstarter.course\_repository)}

\begin{fulllineitems}
\phantomsection\label{cv_kickstarter:cv_kickstarter.course_repository.Course}\pysigline{\strong{class }\code{cv\_kickstarter.course\_repository.}\bfcode{Course}}
Bases: \code{tuple}

Course(title, course\_number, contents, course\_objectives\_text, course\_objectives, tokens)
\index{contents (cv\_kickstarter.course\_repository.Course attribute)}

\begin{fulllineitems}
\phantomsection\label{cv_kickstarter:cv_kickstarter.course_repository.Course.contents}\pysigline{\bfcode{contents}}
Alias for field number 2

\end{fulllineitems}

\index{course\_number (cv\_kickstarter.course\_repository.Course attribute)}

\begin{fulllineitems}
\phantomsection\label{cv_kickstarter:cv_kickstarter.course_repository.Course.course_number}\pysigline{\bfcode{course\_number}}
Alias for field number 1

\end{fulllineitems}

\index{course\_objectives (cv\_kickstarter.course\_repository.Course attribute)}

\begin{fulllineitems}
\phantomsection\label{cv_kickstarter:cv_kickstarter.course_repository.Course.course_objectives}\pysigline{\bfcode{course\_objectives}}
Alias for field number 4

\end{fulllineitems}

\index{course\_objectives\_text (cv\_kickstarter.course\_repository.Course attribute)}

\begin{fulllineitems}
\phantomsection\label{cv_kickstarter:cv_kickstarter.course_repository.Course.course_objectives_text}\pysigline{\bfcode{course\_objectives\_text}}
Alias for field number 3

\end{fulllineitems}

\index{title (cv\_kickstarter.course\_repository.Course attribute)}

\begin{fulllineitems}
\phantomsection\label{cv_kickstarter:cv_kickstarter.course_repository.Course.title}\pysigline{\bfcode{title}}
Alias for field number 0

\end{fulllineitems}

\index{tokens (cv\_kickstarter.course\_repository.Course attribute)}

\begin{fulllineitems}
\phantomsection\label{cv_kickstarter:cv_kickstarter.course_repository.Course.tokens}\pysigline{\bfcode{tokens}}
Alias for field number 5

\end{fulllineitems}


\end{fulllineitems}

\index{CourseRepository (class in cv\_kickstarter.course\_repository)}

\begin{fulllineitems}
\phantomsection\label{cv_kickstarter:cv_kickstarter.course_repository.CourseRepository}\pysiglinewithargsret{\strong{class }\code{cv\_kickstarter.course\_repository.}\bfcode{CourseRepository}}{\emph{mongo\_store}, \emph{collection='courses'}}{}
Bases: \code{object}

Stores and fetches course objects.

CourseRepository is able to fetch course data and deserialize to course
objects and store given course objects by serializing to JSON and
store in MongoDB through the mongo\_store.
\index{create() (cv\_kickstarter.course\_repository.CourseRepository method)}

\begin{fulllineitems}
\phantomsection\label{cv_kickstarter:cv_kickstarter.course_repository.CourseRepository.create}\pysiglinewithargsret{\bfcode{create}}{\emph{course}, \emph{tokens}}{}
Create course with tokens in MongoDB by serializing into JSON.

\end{fulllineitems}

\index{find\_by\_course\_number() (cv\_kickstarter.course\_repository.CourseRepository method)}

\begin{fulllineitems}
\phantomsection\label{cv_kickstarter:cv_kickstarter.course_repository.CourseRepository.find_by_course_number}\pysiglinewithargsret{\bfcode{find\_by\_course\_number}}{\emph{course\_number}}{}
Find course by given course number in MongoDB.

\end{fulllineitems}

\index{remove() (cv\_kickstarter.course\_repository.CourseRepository method)}

\begin{fulllineitems}
\phantomsection\label{cv_kickstarter:cv_kickstarter.course_repository.CourseRepository.remove}\pysiglinewithargsret{\bfcode{remove}}{\emph{course\_number}}{}
Remove course with given course\_number from MongoDB.

\end{fulllineitems}


\end{fulllineitems}

\index{MongoStore (class in cv\_kickstarter.course\_repository)}

\begin{fulllineitems}
\phantomsection\label{cv_kickstarter:cv_kickstarter.course_repository.MongoStore}\pysiglinewithargsret{\strong{class }\code{cv\_kickstarter.course\_repository.}\bfcode{MongoStore}}{\emph{database\_name}, \emph{mongo\_db\_url=None}}{}
Bases: \code{object}

Database Client for MongoDB.
\index{find() (cv\_kickstarter.course\_repository.MongoStore method)}

\begin{fulllineitems}
\phantomsection\label{cv_kickstarter:cv_kickstarter.course_repository.MongoStore.find}\pysiglinewithargsret{\bfcode{find}}{\emph{collection}, \emph{query}}{}
Find document in collection by query.

\end{fulllineitems}

\index{insert() (cv\_kickstarter.course\_repository.MongoStore method)}

\begin{fulllineitems}
\phantomsection\label{cv_kickstarter:cv_kickstarter.course_repository.MongoStore.insert}\pysiglinewithargsret{\bfcode{insert}}{\emph{collection}, \emph{hash\_data}}{}
Insert hash\_data (document) into collection.

\end{fulllineitems}

\index{remove() (cv\_kickstarter.course\_repository.MongoStore method)}

\begin{fulllineitems}
\phantomsection\label{cv_kickstarter:cv_kickstarter.course_repository.MongoStore.remove}\pysiglinewithargsret{\bfcode{remove}}{\emph{collection}, \emph{query}}{}
Remove document in collection by query.

\end{fulllineitems}


\end{fulllineitems}



\section{cv\_kickstarter.cv\_kickstarter\_config module}
\label{cv_kickstarter:cv-kickstarter-cv-kickstarter-config-module}\label{cv_kickstarter:module-cv_kickstarter.cv_kickstarter_config}\index{cv\_kickstarter.cv\_kickstarter\_config (module)}
Configuration for cv\_kickstarter.
\index{CvKickstarterConfig (class in cv\_kickstarter.cv\_kickstarter\_config)}

\begin{fulllineitems}
\phantomsection\label{cv_kickstarter:cv_kickstarter.cv_kickstarter_config.CvKickstarterConfig}\pysiglinewithargsret{\strong{class }\code{cv\_kickstarter.cv\_kickstarter\_config.}\bfcode{CvKickstarterConfig}}{\emph{config\_file\_path='app.cfg'}}{}
Bases: \code{object}

Configuration for cv\_kickstarter.
\index{campus\_net\_app\_name() (cv\_kickstarter.cv\_kickstarter\_config.CvKickstarterConfig method)}

\begin{fulllineitems}
\phantomsection\label{cv_kickstarter:cv_kickstarter.cv_kickstarter_config.CvKickstarterConfig.campus_net_app_name}\pysiglinewithargsret{\bfcode{campus\_net\_app\_name}}{}{}
Return the campus net app name for CampusNet API.

Given by environment variable CAMPUS\_NET\_APP\_NAME or in config file as
{[}campusnet{]}
app\_name: my-app-name

\end{fulllineitems}

\index{campus\_net\_app\_token() (cv\_kickstarter.cv\_kickstarter\_config.CvKickstarterConfig method)}

\begin{fulllineitems}
\phantomsection\label{cv_kickstarter:cv_kickstarter.cv_kickstarter_config.CvKickstarterConfig.campus_net_app_token}\pysiglinewithargsret{\bfcode{campus\_net\_app\_token}}{}{}
Return the campus net app token name for CampusNet API.

Given by environment variable CAMPUS\_NET\_APP\_TOKEN or in config file as
{[}campusnet{]}
app\_token: secret-app-token

\end{fulllineitems}

\index{career\_builder\_key() (cv\_kickstarter.cv\_kickstarter\_config.CvKickstarterConfig method)}

\begin{fulllineitems}
\phantomsection\label{cv_kickstarter:cv_kickstarter.cv_kickstarter_config.CvKickstarterConfig.career_builder_key}\pysiglinewithargsret{\bfcode{career\_builder\_key}}{}{}
Return the career builder api key.

Given by environment variable CAREER\_BUILDER\_DEVELOPER\_KEY or
in config file as
{[}campusnet{]}
app\_token: secret-app-token

\end{fulllineitems}

\index{go\_key() (cv\_kickstarter.cv\_kickstarter\_config.CvKickstarterConfig method)}

\begin{fulllineitems}
\phantomsection\label{cv_kickstarter:cv_kickstarter.cv_kickstarter_config.CvKickstarterConfig.go_key}\pysiglinewithargsret{\bfcode{go\_key}}{}{}
Return the go.dk api key.

Given by environment variable GO\_DEVELOPER\_KEY or
in config file as
{[}godk{]}
guid: secret-guid-key

\end{fulllineitems}

\index{mongo\_db\_name() (cv\_kickstarter.cv\_kickstarter\_config.CvKickstarterConfig method)}

\begin{fulllineitems}
\phantomsection\label{cv_kickstarter:cv_kickstarter.cv_kickstarter_config.CvKickstarterConfig.mongo_db_name}\pysiglinewithargsret{\bfcode{mongo\_db\_name}}{}{}
Return the database name of the Mongo Database.

Given by environment variable MONGO\_DB\_NAME  or in config file as
{[}mongo{]}
db\_name: my\_mongo\_db\_name

\end{fulllineitems}

\index{mongo\_url() (cv\_kickstarter.cv\_kickstarter\_config.CvKickstarterConfig method)}

\begin{fulllineitems}
\phantomsection\label{cv_kickstarter:cv_kickstarter.cv_kickstarter_config.CvKickstarterConfig.mongo_url}\pysiglinewithargsret{\bfcode{mongo\_url}}{}{}
Return the URL for the Mongo DB.

This is not mandatory to get the app to work.

Given by environment variable MONGO\_URL.

\end{fulllineitems}

\index{secret\_key() (cv\_kickstarter.cv\_kickstarter\_config.CvKickstarterConfig method)}

\begin{fulllineitems}
\phantomsection\label{cv_kickstarter:cv_kickstarter.cv_kickstarter_config.CvKickstarterConfig.secret_key}\pysiglinewithargsret{\bfcode{secret\_key}}{}{}
Return the secret key for Flask app.

Given by environment variable SECRET\_KEY or in config file as
{[}flask{]}
secret\_key: my-secret-key

\end{fulllineitems}


\end{fulllineitems}



\section{cv\_kickstarter.dtu\_course\_base module}
\label{cv_kickstarter:module-cv_kickstarter.dtu_course_base}\label{cv_kickstarter:cv-kickstarter-dtu-course-base-module}\index{cv\_kickstarter.dtu\_course\_base (module)}
Structuring courses based on xml response from the DTU Course Base.
\index{Course (class in cv\_kickstarter.dtu\_course\_base)}

\begin{fulllineitems}
\phantomsection\label{cv_kickstarter:cv_kickstarter.dtu_course_base.Course}\pysigline{\strong{class }\code{cv\_kickstarter.dtu\_course\_base.}\bfcode{Course}}
Bases: \code{tuple}

Course(title, course\_number, contents, course\_objectives\_text, course\_objectives)
\index{contents (cv\_kickstarter.dtu\_course\_base.Course attribute)}

\begin{fulllineitems}
\phantomsection\label{cv_kickstarter:cv_kickstarter.dtu_course_base.Course.contents}\pysigline{\bfcode{contents}}
Alias for field number 2

\end{fulllineitems}

\index{course\_number (cv\_kickstarter.dtu\_course\_base.Course attribute)}

\begin{fulllineitems}
\phantomsection\label{cv_kickstarter:cv_kickstarter.dtu_course_base.Course.course_number}\pysigline{\bfcode{course\_number}}
Alias for field number 1

\end{fulllineitems}

\index{course\_objectives (cv\_kickstarter.dtu\_course\_base.Course attribute)}

\begin{fulllineitems}
\phantomsection\label{cv_kickstarter:cv_kickstarter.dtu_course_base.Course.course_objectives}\pysigline{\bfcode{course\_objectives}}
Alias for field number 4

\end{fulllineitems}

\index{course\_objectives\_text (cv\_kickstarter.dtu\_course\_base.Course attribute)}

\begin{fulllineitems}
\phantomsection\label{cv_kickstarter:cv_kickstarter.dtu_course_base.Course.course_objectives_text}\pysigline{\bfcode{course\_objectives\_text}}
Alias for field number 3

\end{fulllineitems}

\index{title (cv\_kickstarter.dtu\_course\_base.Course attribute)}

\begin{fulllineitems}
\phantomsection\label{cv_kickstarter:cv_kickstarter.dtu_course_base.Course.title}\pysigline{\bfcode{title}}
Alias for field number 0

\end{fulllineitems}


\end{fulllineitems}

\index{CourseExtractor (class in cv\_kickstarter.dtu\_course\_base)}

\begin{fulllineitems}
\phantomsection\label{cv_kickstarter:cv_kickstarter.dtu_course_base.CourseExtractor}\pysiglinewithargsret{\strong{class }\code{cv\_kickstarter.dtu\_course\_base.}\bfcode{CourseExtractor}}{\emph{course\_xml}, \emph{language}}{}
Bases: \code{object}

Responsible for extracting course objects based on a course xml.
\index{course() (cv\_kickstarter.dtu\_course\_base.CourseExtractor method)}

\begin{fulllineitems}
\phantomsection\label{cv_kickstarter:cv_kickstarter.dtu_course_base.CourseExtractor.course}\pysiglinewithargsret{\bfcode{course}}{}{}
Return a Course object based on the given course xml.

\end{fulllineitems}


\end{fulllineitems}

\index{courses\_from\_xml() (in module cv\_kickstarter.dtu\_course\_base)}

\begin{fulllineitems}
\phantomsection\label{cv_kickstarter:cv_kickstarter.dtu_course_base.courses_from_xml}\pysiglinewithargsret{\code{cv\_kickstarter.dtu\_course\_base.}\bfcode{courses\_from\_xml}}{\emph{courses\_xml\_text}, \emph{language='en-GB'}}{}
Extract course objects based on xml with courses.

\end{fulllineitems}



\section{cv\_kickstarter.dtu\_skill\_set module}
\label{cv_kickstarter:module-cv_kickstarter.dtu_skill_set}\label{cv_kickstarter:cv-kickstarter-dtu-skill-set-module}\index{cv\_kickstarter.dtu\_skill\_set (module)}
Extract the skill set of a DTU student.

This module is a layer on top of academic\_skill\_set that takes exam results
as given by the CampusNet API, merges it with course information from
the Course Base (through the course base repo).
\index{CampusNetCourseBaseMerger (class in cv\_kickstarter.dtu\_skill\_set)}

\begin{fulllineitems}
\phantomsection\label{cv_kickstarter:cv_kickstarter.dtu_skill_set.CampusNetCourseBaseMerger}\pysiglinewithargsret{\strong{class }\code{cv\_kickstarter.dtu\_skill\_set.}\bfcode{CampusNetCourseBaseMerger}}{\emph{exam\_result\_programmes}, \emph{course\_base}}{}
Bases: \code{object}

Merges exam results with information about the course.
\index{course\_exam\_results() (cv\_kickstarter.dtu\_skill\_set.CampusNetCourseBaseMerger method)}

\begin{fulllineitems}
\phantomsection\label{cv_kickstarter:cv_kickstarter.dtu_skill_set.CampusNetCourseBaseMerger.course_exam_results}\pysiglinewithargsret{\bfcode{course\_exam\_results}}{}{}
Return course exam results.

A list of ExamResult, that contains exam result information and
more detailed information about the course given by the returned
course object from course base.

\end{fulllineitems}


\end{fulllineitems}

\index{DtuSkillSet (class in cv\_kickstarter.dtu\_skill\_set)}

\begin{fulllineitems}
\phantomsection\label{cv_kickstarter:cv_kickstarter.dtu_skill_set.DtuSkillSet}\pysiglinewithargsret{\strong{class }\code{cv\_kickstarter.dtu\_skill\_set.}\bfcode{DtuSkillSet}}{\emph{exam\_result\_programmes}, \emph{course\_base\_repo}}{}
Bases: \code{object}

Responsible for extracting a skill set based on DTU data sources.
\index{skill\_set() (cv\_kickstarter.dtu\_skill\_set.DtuSkillSet method)}

\begin{fulllineitems}
\phantomsection\label{cv_kickstarter:cv_kickstarter.dtu_skill_set.DtuSkillSet.skill_set}\pysiglinewithargsret{\bfcode{skill\_set}}{}{}
Return the skill set.

\end{fulllineitems}


\end{fulllineitems}

\index{ExamResult (class in cv\_kickstarter.dtu\_skill\_set)}

\begin{fulllineitems}
\phantomsection\label{cv_kickstarter:cv_kickstarter.dtu_skill_set.ExamResult}\pysiglinewithargsret{\strong{class }\code{cv\_kickstarter.dtu\_skill\_set.}\bfcode{ExamResult}}{\emph{grade}, \emph{course}, \emph{ects\_points}}{}
Bases: \code{object}

Exam result with grade, course and ects points.
\index{course\_tokens (cv\_kickstarter.dtu\_skill\_set.ExamResult attribute)}

\begin{fulllineitems}
\phantomsection\label{cv_kickstarter:cv_kickstarter.dtu_skill_set.ExamResult.course_tokens}\pysigline{\bfcode{course\_tokens}}
Return the tokens of the course.

\end{fulllineitems}


\end{fulllineitems}

\index{TokenizedCourseExamResult (class in cv\_kickstarter.dtu\_skill\_set)}

\begin{fulllineitems}
\phantomsection\label{cv_kickstarter:cv_kickstarter.dtu_skill_set.TokenizedCourseExamResult}\pysigline{\strong{class }\code{cv\_kickstarter.dtu\_skill\_set.}\bfcode{TokenizedCourseExamResult}}
Bases: \code{tuple}

TokenizedCourseExamResult(exam\_result, tokens, course)
\index{course (cv\_kickstarter.dtu\_skill\_set.TokenizedCourseExamResult attribute)}

\begin{fulllineitems}
\phantomsection\label{cv_kickstarter:cv_kickstarter.dtu_skill_set.TokenizedCourseExamResult.course}\pysigline{\bfcode{course}}
Alias for field number 2

\end{fulllineitems}

\index{exam\_result (cv\_kickstarter.dtu\_skill\_set.TokenizedCourseExamResult attribute)}

\begin{fulllineitems}
\phantomsection\label{cv_kickstarter:cv_kickstarter.dtu_skill_set.TokenizedCourseExamResult.exam_result}\pysigline{\bfcode{exam\_result}}
Alias for field number 0

\end{fulllineitems}

\index{tokens (cv\_kickstarter.dtu\_skill\_set.TokenizedCourseExamResult attribute)}

\begin{fulllineitems}
\phantomsection\label{cv_kickstarter:cv_kickstarter.dtu_skill_set.TokenizedCourseExamResult.tokens}\pysigline{\bfcode{tokens}}
Alias for field number 1

\end{fulllineitems}


\end{fulllineitems}



\section{cv\_kickstarter.ects\_grade\_calculator module}
\label{cv_kickstarter:cv-kickstarter-ects-grade-calculator-module}\label{cv_kickstarter:module-cv_kickstarter.ects_grade_calculator}\index{cv\_kickstarter.ects\_grade\_calculator (module)}
Calculates average grade based on grades weighted by ECTS-point.
\index{GradeAverageCalculator (class in cv\_kickstarter.ects\_grade\_calculator)}

\begin{fulllineitems}
\phantomsection\label{cv_kickstarter:cv_kickstarter.ects_grade_calculator.GradeAverageCalculator}\pysiglinewithargsret{\strong{class }\code{cv\_kickstarter.ects\_grade\_calculator.}\bfcode{GradeAverageCalculator}}{\emph{exam\_results}}{}
Bases: \code{object}

Class that is able to calculate average based on ECTS exam\_results.
\index{average\_grade() (cv\_kickstarter.ects\_grade\_calculator.GradeAverageCalculator method)}

\begin{fulllineitems}
\phantomsection\label{cv_kickstarter:cv_kickstarter.ects_grade_calculator.GradeAverageCalculator.average_grade}\pysiglinewithargsret{\bfcode{average\_grade}}{}{}
Return the average grade.

\end{fulllineitems}


\end{fulllineitems}

\index{average\_grade() (in module cv\_kickstarter.ects\_grade\_calculator)}

\begin{fulllineitems}
\phantomsection\label{cv_kickstarter:cv_kickstarter.ects_grade_calculator.average_grade}\pysiglinewithargsret{\code{cv\_kickstarter.ects\_grade\_calculator.}\bfcode{average\_grade}}{\emph{exam\_results}}{}
Return the average grade from a list of exam\_results.

\end{fulllineitems}



\section{cv\_kickstarter.nltk\_data\_downloader module}
\label{cv_kickstarter:cv-kickstarter-nltk-data-downloader-module}\label{cv_kickstarter:module-cv_kickstarter.nltk_data_downloader}\index{cv\_kickstarter.nltk\_data\_downloader (module)}
Downloads nltk data relevant for the cv kickstarter project.

If any new module needs data from nltk, it should be added here.
\index{download() (in module cv\_kickstarter.nltk\_data\_downloader)}

\begin{fulllineitems}
\phantomsection\label{cv_kickstarter:cv_kickstarter.nltk_data_downloader.download}\pysiglinewithargsret{\code{cv\_kickstarter.nltk\_data\_downloader.}\bfcode{download}}{}{}
Download relevant nltk data for cv kickstarter.

If there already exists a nltk\_data folder in the root, it is not
downloaded.

If nltk data is added the nltk\_data folder need to be removed for
downloading the new data.

\end{fulllineitems}



\section{cv\_kickstarter.session\_authentication module}
\label{cv_kickstarter:module-cv_kickstarter.session_authentication}\label{cv_kickstarter:cv-kickstarter-session-authentication-module}\index{cv\_kickstarter.session\_authentication (module)}
Utilizes the a session dictionary as authentication storage.

Integrates with the Flask session.
\index{SessionAuthentication (class in cv\_kickstarter.session\_authentication)}

\begin{fulllineitems}
\phantomsection\label{cv_kickstarter:cv_kickstarter.session_authentication.SessionAuthentication}\pysiglinewithargsret{\strong{class }\code{cv\_kickstarter.session\_authentication.}\bfcode{SessionAuthentication}}{\emph{session\_dict}}{}
Bases: \code{object}

Utilizes the a session dictionary as authentication storage.
\index{auth\_token (cv\_kickstarter.session\_authentication.SessionAuthentication attribute)}

\begin{fulllineitems}
\phantomsection\label{cv_kickstarter:cv_kickstarter.session_authentication.SessionAuthentication.auth_token}\pysigline{\bfcode{auth\_token}}
Return authentication token from session.

\end{fulllineitems}

\index{authenticate() (cv\_kickstarter.session\_authentication.SessionAuthentication method)}

\begin{fulllineitems}
\phantomsection\label{cv_kickstarter:cv_kickstarter.session_authentication.SessionAuthentication.authenticate}\pysiglinewithargsret{\bfcode{authenticate}}{\emph{student\_id}, \emph{auth\_token}}{}
Authenticate a student by setting student\_id and auth\_token.

\end{fulllineitems}

\index{is\_authenticated() (cv\_kickstarter.session\_authentication.SessionAuthentication method)}

\begin{fulllineitems}
\phantomsection\label{cv_kickstarter:cv_kickstarter.session_authentication.SessionAuthentication.is_authenticated}\pysiglinewithargsret{\bfcode{is\_authenticated}}{}{}
Return true if the user is authenticated in the session.

\end{fulllineitems}

\index{log\_out() (cv\_kickstarter.session\_authentication.SessionAuthentication method)}

\begin{fulllineitems}
\phantomsection\label{cv_kickstarter:cv_kickstarter.session_authentication.SessionAuthentication.log_out}\pysiglinewithargsret{\bfcode{log\_out}}{}{}
Log the user out by removing student\_id and auth\_token.

\end{fulllineitems}

\index{student\_id (cv\_kickstarter.session\_authentication.SessionAuthentication attribute)}

\begin{fulllineitems}
\phantomsection\label{cv_kickstarter:cv_kickstarter.session_authentication.SessionAuthentication.student_id}\pysigline{\bfcode{student\_id}}
Return student id from session.

\end{fulllineitems}


\end{fulllineitems}



\section{Module contents}
\label{cv_kickstarter:module-contents}\label{cv_kickstarter:module-cv_kickstarter}\index{cv\_kickstarter (module)}
CVKickstarer - generate CV with skills and job recommmendations.

CVKickstarter is a web app that is able to analyse your DTU information
and provide you with a prefilled CV with your education information, courses
you've passed, the performance of your study, the skills you've gained and
job suggestions based on your best skills.


\chapter{job\_searcher package}
\label{job_searcher:job-searcher-package}\label{job_searcher::doc}

\section{Submodules}
\label{job_searcher:submodules}

\section{job\_searcher.career\_builder module}
\label{job_searcher:module-job_searcher.career_builder}\label{job_searcher:job-searcher-career-builder-module}\index{job\_searcher.career\_builder (module)}
Module for job searching on CareerBuilder.com.
\index{CareerBuilder (class in job\_searcher.career\_builder)}

\begin{fulllineitems}
\phantomsection\label{job_searcher:job_searcher.career_builder.CareerBuilder}\pysiglinewithargsret{\strong{class }\code{job\_searcher.career\_builder.}\bfcode{CareerBuilder}}{\emph{developer\_key}}{}
Bases: \code{job\_searcher.job\_searcher.JobSearcher}

JobSearcher for CareerBuilder.com.
\index{BASE\_URL (job\_searcher.career\_builder.CareerBuilder attribute)}

\begin{fulllineitems}
\phantomsection\label{job_searcher:job_searcher.career_builder.CareerBuilder.BASE_URL}\pysigline{\bfcode{BASE\_URL}\strong{ = `http://api.careerbuilder.com/v2/jobsearch'}}
\end{fulllineitems}

\index{PARAM\_DEV\_KEY (job\_searcher.career\_builder.CareerBuilder attribute)}

\begin{fulllineitems}
\phantomsection\label{job_searcher:job_searcher.career_builder.CareerBuilder.PARAM_DEV_KEY}\pysigline{\bfcode{PARAM\_DEV\_KEY}\strong{ = `DeveloperKey'}}
\end{fulllineitems}

\index{PARAM\_KEYWORDS (job\_searcher.career\_builder.CareerBuilder attribute)}

\begin{fulllineitems}
\phantomsection\label{job_searcher:job_searcher.career_builder.CareerBuilder.PARAM_KEYWORDS}\pysigline{\bfcode{PARAM\_KEYWORDS}\strong{ = `keywords'}}
\end{fulllineitems}

\index{PARAM\_PER\_PAGE (job\_searcher.career\_builder.CareerBuilder attribute)}

\begin{fulllineitems}
\phantomsection\label{job_searcher:job_searcher.career_builder.CareerBuilder.PARAM_PER_PAGE}\pysigline{\bfcode{PARAM\_PER\_PAGE}\strong{ = `perpage'}}
\end{fulllineitems}

\index{find\_results() (job\_searcher.career\_builder.CareerBuilder method)}

\begin{fulllineitems}
\phantomsection\label{job_searcher:job_searcher.career_builder.CareerBuilder.find_results}\pysiglinewithargsret{\bfcode{find\_results}}{\emph{keywords={[}{]}}, \emph{amount=5}}{}
Perform a job search.
\begin{quote}\begin{description}
\item[{Parameters}] \leavevmode
\textbf{keywords} -- Keywords, that should be contained in the returned

\end{description}\end{quote}

results.
:param amount: The amount of results wanted.
:return: The jobs found by the given search parameters.

\end{fulllineitems}

\index{find\_results\_amount() (job\_searcher.career\_builder.CareerBuilder method)}

\begin{fulllineitems}
\phantomsection\label{job_searcher:job_searcher.career_builder.CareerBuilder.find_results_amount}\pysiglinewithargsret{\bfcode{find\_results\_amount}}{\emph{keyword='`}}{}
Find the amount of results for a given keyword.

\end{fulllineitems}

\index{xml\_to\_jobs() (job\_searcher.career\_builder.CareerBuilder static method)}

\begin{fulllineitems}
\phantomsection\label{job_searcher:job_searcher.career_builder.CareerBuilder.xml_to_jobs}\pysiglinewithargsret{\strong{static }\bfcode{xml\_to\_jobs}}{\emph{xml}}{}
Convert xml to a list of jobs.
\begin{quote}\begin{description}
\item[{Parameters}] \leavevmode
\textbf{xml} -- The xml string, that should be parsed.

\item[{Returns}] \leavevmode
A list of jobs.

\end{description}\end{quote}

\end{fulllineitems}


\end{fulllineitems}



\section{job\_searcher.go\_jobs module}
\label{job_searcher:module-job_searcher.go_jobs}\label{job_searcher:job-searcher-go-jobs-module}\index{job\_searcher.go\_jobs (module)}
Module for job searching on Go.dk.
\index{GoJobs (class in job\_searcher.go\_jobs)}

\begin{fulllineitems}
\phantomsection\label{job_searcher:job_searcher.go_jobs.GoJobs}\pysiglinewithargsret{\strong{class }\code{job\_searcher.go\_jobs.}\bfcode{GoJobs}}{\emph{guid}}{}
Bases: \code{job\_searcher.job\_searcher.JobSearcher}

JobSearcher for Go.dk.
\index{BASE\_URL (job\_searcher.go\_jobs.GoJobs attribute)}

\begin{fulllineitems}
\phantomsection\label{job_searcher:job_searcher.go_jobs.GoJobs.BASE_URL}\pysigline{\bfcode{BASE\_URL}\strong{ = u'http://moveon.dk/webservice/mobile.asmx/'}}
\end{fulllineitems}

\index{HEADERS (job\_searcher.go\_jobs.GoJobs attribute)}

\begin{fulllineitems}
\phantomsection\label{job_searcher:job_searcher.go_jobs.GoJobs.HEADERS}\pysigline{\bfcode{HEADERS}\strong{ = \{u'Content-type': u'application/json', u'Accept': u'text/plain'\}}}
\end{fulllineitems}

\index{PASS (job\_searcher.go\_jobs.GoJobs attribute)}

\begin{fulllineitems}
\phantomsection\label{job_searcher:job_searcher.go_jobs.GoJobs.PASS}\pysigline{\bfcode{PASS}\strong{ = u`02e19abe-b6f4-4a7e-bb70-9e613fcb43c2'}}
\end{fulllineitems}

\index{URL\_EXTENSION\_GET\_JOB (job\_searcher.go\_jobs.GoJobs attribute)}

\begin{fulllineitems}
\phantomsection\label{job_searcher:job_searcher.go_jobs.GoJobs.URL_EXTENSION_GET_JOB}\pysigline{\bfcode{URL\_EXTENSION\_GET\_JOB}\strong{ = u'GetJobLimitedV3'}}
\end{fulllineitems}

\index{URL\_EXTENSION\_SEARCH (job\_searcher.go\_jobs.GoJobs attribute)}

\begin{fulllineitems}
\phantomsection\label{job_searcher:job_searcher.go_jobs.GoJobs.URL_EXTENSION_SEARCH}\pysigline{\bfcode{URL\_EXTENSION\_SEARCH}\strong{ = u'SearchJobsV3'}}
\end{fulllineitems}

\index{find\_results() (job\_searcher.go\_jobs.GoJobs method)}

\begin{fulllineitems}
\phantomsection\label{job_searcher:job_searcher.go_jobs.GoJobs.find_results}\pysiglinewithargsret{\bfcode{find\_results}}{\emph{keywords=()}, \emph{amount=5}}{}
Perform a job search.
\begin{quote}\begin{description}
\item[{Parameters}] \leavevmode
\textbf{keywords} -- Keywords, that should be contained in the returned

\end{description}\end{quote}

results.
:param amount: The amount of results wanted.
:return: The jobs found by the given search parameters.

\end{fulllineitems}

\index{find\_results\_amount() (job\_searcher.go\_jobs.GoJobs method)}

\begin{fulllineitems}
\phantomsection\label{job_searcher:job_searcher.go_jobs.GoJobs.find_results_amount}\pysiglinewithargsret{\bfcode{find\_results\_amount}}{\emph{keyword=u'`}}{}
Find the amount of results for a given keyword.

\end{fulllineitems}

\index{get\_details\_for\_job() (job\_searcher.go\_jobs.GoJobs method)}

\begin{fulllineitems}
\phantomsection\label{job_searcher:job_searcher.go_jobs.GoJobs.get_details_for_job}\pysiglinewithargsret{\bfcode{get\_details\_for\_job}}{\emph{job\_id}}{}
Get job details for the given job id.
\begin{quote}\begin{description}
\item[{Parameters}] \leavevmode
\textbf{job\_id} -- Id for the job, that you want to retrieve.

\item[{Returns}] \leavevmode
A job.

\end{description}\end{quote}

\end{fulllineitems}

\index{get\_details\_for\_jobs() (job\_searcher.go\_jobs.GoJobs method)}

\begin{fulllineitems}
\phantomsection\label{job_searcher:job_searcher.go_jobs.GoJobs.get_details_for_jobs}\pysiglinewithargsret{\bfcode{get\_details\_for\_jobs}}{\emph{job\_ids}}{}
Get job details for the given job id's.
\begin{quote}\begin{description}
\item[{Parameters}] \leavevmode
\textbf{job\_ids} -- Id's for the jobs, that you want to retrieve.

\item[{Returns}] \leavevmode
A list of the given jobs.

\end{description}\end{quote}

\end{fulllineitems}

\index{json\_to\_job() (job\_searcher.go\_jobs.GoJobs static method)}

\begin{fulllineitems}
\phantomsection\label{job_searcher:job_searcher.go_jobs.GoJobs.json_to_job}\pysiglinewithargsret{\strong{static }\bfcode{json\_to\_job}}{\emph{json\_text}}{}
Convert a json string to a job.
\begin{quote}\begin{description}
\item[{Parameters}] \leavevmode
\textbf{json\_text} -- The json string, that should be parsed.

\item[{Returns}] \leavevmode
A job.

\end{description}\end{quote}

\end{fulllineitems}


\end{fulllineitems}



\section{job\_searcher.job module}
\label{job_searcher:module-job_searcher.job}\label{job_searcher:job-searcher-job-module}\index{job\_searcher.job (module)}
Generic job module.
\index{Job (class in job\_searcher.job)}

\begin{fulllineitems}
\phantomsection\label{job_searcher:job_searcher.job.Job}\pysiglinewithargsret{\strong{class }\code{job\_searcher.job.}\bfcode{Job}}{\emph{title}, \emph{company\_name}, \emph{teaser}, \emph{job\_url}}{}
Generic Job interface.

Simple job class, used to expose a generic interface for jobs
across different job searchers.

\end{fulllineitems}



\section{job\_searcher.keyword\_evaluator module}
\label{job_searcher:module-job_searcher.keyword_evaluator}\label{job_searcher:job-searcher-keyword-evaluator-module}\index{job\_searcher.keyword\_evaluator (module)}
A module used for keyword evalution.
\index{KeywordEvaluator (class in job\_searcher.keyword\_evaluator)}

\begin{fulllineitems}
\phantomsection\label{job_searcher:job_searcher.keyword_evaluator.KeywordEvaluator}\pysiglinewithargsret{\strong{class }\code{job\_searcher.keyword\_evaluator.}\bfcode{KeywordEvaluator}}{\emph{job\_searcher}}{}
A class used to evaluate keywords with a given job searcher.
\index{evaluate\_keyword() (job\_searcher.keyword\_evaluator.KeywordEvaluator method)}

\begin{fulllineitems}
\phantomsection\label{job_searcher:job_searcher.keyword_evaluator.KeywordEvaluator.evaluate_keyword}\pysiglinewithargsret{\bfcode{evaluate\_keyword}}{\emph{keyword}}{}
Evaluate a keyword for the given job searcher.
\begin{quote}\begin{description}
\item[{Parameters}] \leavevmode
\textbf{keyword} -- The keyword to evaluate.

\item[{Returns}] \leavevmode
The percentage of jobs, that contain the given keyword.

\end{description}\end{quote}

\end{fulllineitems}


\end{fulllineitems}



\section{Module contents}
\label{job_searcher:module-job_searcher}\label{job_searcher:module-contents}\index{job\_searcher (module)}
Job searching module for searching jobs and evaluate skill relevanse.


\chapter{Using CV Kickstarter}
\label{index:using-cv-kickstarter}
Data Mining using Python project repository


\section{Setup}
\label{index:setup}
To install required modules, simply type:

\begin{Verbatim}[commandchars=\\\{\}]
python setup.py install
\end{Verbatim}

For optimization reasons, the course information is fetched from a MongoDB database. In order to import courses into MongoDB from the xml, run the command:

\begin{Verbatim}[commandchars=\\\{\}]
python course\PYGZus{}import.py
\end{Verbatim}


\section{Configuration}
\label{index:configuration}
To configure the application, it is possible to either add an app.cfg file that contains the relevant configurations (see app.cfg.example for an example configuration file) or environment variables (see \emph{cv\_kickstarter\_config.py} for environment variables used).

The app defaults to using MongoDB through localhost unless a \emph{MONGO\_URL} environment variable is given.


\section{Run webserver}
\label{index:run-webserver}
To run the web server after the setup, execute:

\begin{Verbatim}[commandchars=\\\{\}]
python webapp.py
\end{Verbatim}

or to run with gunicorn, execute:

\begin{Verbatim}[commandchars=\\\{\}]
gunicorn webapp:app \PYGZhy{}\PYGZhy{}log\PYGZhy{}file=\PYGZhy{}
\end{Verbatim}


\section{Commmand Line Integration}
\label{index:commmand-line-integration}
The CV can also be exported in a json format through the CLI by using the command:

\begin{Verbatim}[commandchars=\\\{\}]
jsoncv s123456 secret
\end{Verbatim}

where `s123456' should be your student id and `secret' be your password to CampusNet.


\section{Tests}
\label{index:tests}
The project uses \code{py.test} for testing, so to run tests, execute:

\begin{Verbatim}[commandchars=\\\{\}]
\PYG{n}{py}\PYG{o}{.}\PYG{n}{test}
\end{Verbatim}

The project is also set up with \code{tox} and is tested against python
2.7, 3.4, pypy and pypy3. To run tox locally, type:

\begin{Verbatim}[commandchars=\\\{\}]
python setup.py test
\end{Verbatim}


\chapter{Indices and tables}
\label{index:indices-and-tables}\begin{itemize}
\item {} 
\emph{genindex}

\item {} 
\emph{modindex}

\item {} 
\emph{search}

\end{itemize}


\renewcommand{\indexname}{Python Module Index}
\begin{theindex}
\def\bigletter#1{{\Large\sffamily#1}\nopagebreak\vspace{1mm}}
\bigletter{c}
\item {\texttt{cv\_kickstarter}}, \pageref{cv_kickstarter:module-cv_kickstarter}
\item {\texttt{cv\_kickstarter.academic\_skill\_set}}, \pageref{cv_kickstarter:module-cv_kickstarter.academic_skill_set}
\item {\texttt{cv\_kickstarter.cnapi}}, \pageref{cv_kickstarter:module-cv_kickstarter.cnapi}
\item {\texttt{cv\_kickstarter.course\_keyword\_tokenizer}}, \pageref{cv_kickstarter:module-cv_kickstarter.course_keyword_tokenizer}
\item {\texttt{cv\_kickstarter.course\_repository}}, \pageref{cv_kickstarter:module-cv_kickstarter.course_repository}
\item {\texttt{cv\_kickstarter.cv\_kickstarter\_config}}, \pageref{cv_kickstarter:module-cv_kickstarter.cv_kickstarter_config}
\item {\texttt{cv\_kickstarter.dtu\_course\_base}}, \pageref{cv_kickstarter:module-cv_kickstarter.dtu_course_base}
\item {\texttt{cv\_kickstarter.dtu\_skill\_set}}, \pageref{cv_kickstarter:module-cv_kickstarter.dtu_skill_set}
\item {\texttt{cv\_kickstarter.ects\_grade\_calculator}}, \pageref{cv_kickstarter:module-cv_kickstarter.ects_grade_calculator}
\item {\texttt{cv\_kickstarter.models}}, \pageref{cv_kickstarter.models:module-cv_kickstarter.models}
\item {\texttt{cv\_kickstarter.models.exam\_result\_programme}}, \pageref{cv_kickstarter.models:module-cv_kickstarter.models.exam_result_programme}
\item {\texttt{cv\_kickstarter.models.user\_cv}}, \pageref{cv_kickstarter.models:module-cv_kickstarter.models.user_cv}
\item {\texttt{cv\_kickstarter.models.user\_cv\_builder}}, \pageref{cv_kickstarter.models:module-cv_kickstarter.models.user_cv_builder}
\item {\texttt{cv\_kickstarter.models.user\_cv\_dictionary\_mapper}}, \pageref{cv_kickstarter.models:module-cv_kickstarter.models.user_cv_dictionary_mapper}
\item {\texttt{cv\_kickstarter.nltk\_data\_downloader}}, \pageref{cv_kickstarter:module-cv_kickstarter.nltk_data_downloader}
\item {\texttt{cv\_kickstarter.session\_authentication}}, \pageref{cv_kickstarter:module-cv_kickstarter.session_authentication}
\indexspace
\bigletter{j}
\item {\texttt{job\_searcher}}, \pageref{job_searcher:module-job_searcher}
\item {\texttt{job\_searcher.career\_builder}}, \pageref{job_searcher:module-job_searcher.career_builder}
\item {\texttt{job\_searcher.go\_jobs}}, \pageref{job_searcher:module-job_searcher.go_jobs}
\item {\texttt{job\_searcher.job}}, \pageref{job_searcher:module-job_searcher.job}
\item {\texttt{job\_searcher.keyword\_evaluator}}, \pageref{job_searcher:module-job_searcher.keyword_evaluator}
\end{theindex}

\renewcommand{\indexname}{Index}
\printindex
\end{document}
